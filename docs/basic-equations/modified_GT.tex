\documentclass[12pt]{article}
%\usepackage{helvet}
%\renewcommand{\familydefault}{\sfdefault}
\usepackage{amsfonts}
\usepackage{amsmath}
\usepackage{amssymb}
\usepackage{bm}
\usepackage{fullpage}
\usepackage{setspace}
\usepackage{graphicx}
\usepackage{gensymb}
\usepackage[nottoc,numbib]{tocbibind}
\usepackage{graphicx}
\usepackage{float}
\usepackage{braket}
\usepackage{titlesec}
\usepackage{siunitx}
\usepackage{mathtools}
\usepackage{tikz}
\usepackage[font={small}]{caption}
\usepackage{subcaption}
%\titlespacing*{\section}{0pt}{4pt}{4pt}
\usepackage[letterpaper, margin=2cm]{geometry}
\newcommand*\diff{\mathop{}\!\mathrm{d}}
\DeclareSIUnit\Molar{\textsc{m}}
\DeclareMathOperator*{\MAXVAL}{max}
\DeclarePairedDelimiter\abs{\lvert}{\rvert}
\tikzstyle{place}=[circle,draw,fill,thick,inner sep = 0pt,minimum size = 1.5mm]
\tikzstyle{theline}=[line width = 1pt]


\begin{document}
\pagenumbering{arabic}
\spacing{1.5}

%%%%%%%%%%%
% Begin Document
%%%%%%%%%%%
\section{Generalized Gibbs-Thomson boundary condition}
This derivation follows section 8.3.3 of \cite{Doi_Ch8}. My goal is to derive the boundary condition between a droplet of A-rich solution with concentration $c_a$ and radius $R$ in a sea of B-rich solution with concentration $c_b$. Note that I am assuming that $\phi(\bm{r},t)$ does not change anymore in the domains (it stays fixed at $c_a$ and $c_b$ within the A-rich and B-rich phases, respectively), only the size of the domains change. This assumption requires us to be working in the late-stage regime of phase separation. A schematic of the problem for a single droplet is shown in Figure \ref{fig:single_drop}.

\begin{figure}[H]
	\centering
	\begin{subfigure}[t]{0.45\textwidth}
		\includegraphics[width=\textwidth]{"A_in_B_sea".png}
		\caption{}
		\label{fig:ABsea}
	\end{subfigure}\qquad%
	\begin{subfigure}[t]{0.45\textwidth}
		\includegraphics[width=\textwidth]{"R_independent_f".png}
		\caption{}
		\label{fig:f_phi}
	\end{subfigure}
	\caption{(\protect\subref{fig:ABsea}) Geometry of the problem considered in the limit of only a single ($d$-dimensional) spherical droplet of A-rich phase in a sea of B-rich phase. (\protect\subref{fig:f_phi}) Typical form of the free energy density when no internal structure of the phases are taken into account. Taken from \cite{Doi_Ch8}.}
\end{figure}\label{fig:single_drop}

When a droplet has a non-constant bulk free energy per unit volume for a given radius, it must be taken into account when considering it's growth dynamics. We are considering the case in which our droplet is a non-cross-linked liquid crystal (surface) phase, with a total volume of a $d$-dimensional sphere. Molecules from the B-rich (A-rich) phase are free to diffuse into (out of) the liquid crystal phase, increasing (decreasing) the radius $R$.

Suppose that the bulk free energy of a single droplet of radius $R$ and concentration $c_a$ is $F_d(R,c_a)$ (for a homogeneous liquid droplet, $F_d(R,c_a)=v_dR^df(c_a)$, but in general can have a more intricate dependence on $R$). If the free energy density of the surrounding majority phase with concentration $c_b$ is $f(c_b)$, then for $N$ droplets, the total free energy of the system (droplets + majority phase solution) is
\begin{equation}\label{eq:FE1}
F=NF_d+(V-Nv_dR^d)f(c_b)+Na_dR^{d-1}\gamma,
\end{equation}
where $\gamma$ is the surface tension between a droplet and the solution. To simplify notation, I'll include the surface term in droplet free energy, and define
\begin{equation}\label{eq:FE2}
F=Nv_dR^dE(R,c_a)+(V-Nv_dR^d)f(c_b),
\end{equation}
and
\begin{equation}\label{eq:E_R}
E(R,c_a)=\frac{1}{v_dR^d}\left[F_d(R,c_a)+a_dR^{d-1}\gamma\right]
\end{equation}
is the free energy per unit volume of a single droplet. Eqn \ref{eq:FE2} must be minimized subject to the constraint
\begin{equation}\label{eq:N_s}
g(R,N,c_a,c_b)=Nv_dR^dc_a+(V-Nv_dR^d)c_b-N_s=0,
\end{equation}
which states that the solute ($A$-component) number $N_s$ must remain constant in the system. For the system in equilibrium, the constrained free energy is a minimum, i.e.
\begin{align}
&\frac{\partial [F-\lambda g(R,n,c_a,c_b)]}{\partial c_a} = 0\label{eq:min_wrt_a},\\
&\frac{\partial [F-\lambda g(R,n,c_a,c_b)]}{\partial c_b} = 0\label{eq:min_wrt_b},\\
&\frac{\partial [F-\lambda g(R,n,c_a,c_b)]}{\partial R} = 0\label{eq:min_wrt_R},\\
&\frac{\partial [F-\lambda g(R,n,c_a,c_b)]}{\partial N} = 0\label{eq:min_wrt_N}.\\
\end{align}
The first two equations give the result
\begin{equation}\label{eq:phi_mins_eq}
\lambda=\frac{\partial E}{\partial c_a}=\frac{\partial f}{\partial c_b}.
\end{equation}
The minimization with respect to $R$ (and using eqn \ref{eq:phi_mins_eq}) results in the equation
\begin{equation}\label{eq:R_min_eq}
\frac{\partial f}{\partial c_b}=\frac{1}{c_a-c_b}\left[E(R,c_a)+\frac{R}{d}\frac{\partial E}{\partial R}-f(c_b)\right].
\end{equation}
Through the minimization of the number of fibrils, eqns \ref{eq:min_wrt_a}, \ref{eq:min_wrt_b}, and \ref{eq:min_wrt_N} give the final equation
\begin{equation}\label{eq:N_min_eq}
\frac{\partial f}{\partial c_b}=\frac{1}{c_a-c_b}\left[E(R,c_a)-f(c_b)\right].
\end{equation}

Solving these equations gives the expected result that the equilibrium droplet distribution is given by the condition
\begin{equation}\label{eq:bulk_FE0}
\left.\frac{\partial E}{\partial R}\right|_{R=R_{\text{eq}}}=0,
\end{equation}
where $R_{\text{eq}}$ is the equilibrium droplet size.
and the concentrations within the two phases $c_a$ and $c_b$, as well as the number of droplets in the system per unit volume is given by eqns \ref{eq:phi_mins_eq}, \ref{eq:N_min_eq}, along with the constraint eqn \ref{eq:N_s} (for a prescribed number of solute molecules $N_s$ and system size $V$).

\section{Quadratic free energy Gibbs-Thomson}
For the purposes of the simulation, we want to determine the Gibbs-Thomson boundary condition for a given number of droplets, and have the dynamics drive the system to $n_{\text{eq}}$. Therefore, we hold $n$ constant when minimizing the constrained free energy, which means the condition of eqn \ref{eq:n_min_eq} is no applicable. The change in concentration near each droplet from being out of equilibrium is then
\begin{align}\label{eq:GT}
\delta c_b&=\frac{1}{(c_a-c_b)f''(c_b)}\left[E(R,c_a)+\frac{R}{d}\frac{\partial E}{\partial R}-E(R_{\text{eq}},c_a)-\left.\frac{\partial E}{\partial R}\right|_{R=R_{\text{eq}}}\right]\nonumber\\
&=\frac{1}{(c_a-c_b)f''(c_b)}\left[E(R,c_a)+\frac{R}{d}\frac{\partial E}{\partial R}-E(R_{\text{eq}},c_a)\right].
\end{align}

If I take the free energy per unit volume of the system to be
\begin{equation}\label{eq:E_per_V}
E(R,c_a)=E(R_{\text{eq}},c_a)+\frac{1}{2}\left.\frac{\partial^2 E}{\partial R^2}\right|_{R=R_{\text{eq}}}(R-R_{\text{eq}})^2
\end{equation}
(compared to the standard, liquid droplet form of $E_l(R)=f(c_a)+a_d\gamma (v_dR)^{-1}$), then the Gibbs-Thomson boundary condition becomes
\begin{equation}\label{eq:GT_quad_dim}
\delta c_b=\frac{E''(R_{\text{eq}})}{2(c_a-c_b)f''(c_b)}\left(\frac{d+2}{d}R-R_{\text{eq}}\right)(R-R_{\text{eq}}).
\end{equation}
To get this into the dimensionless form necessary for the calculations, I start off by defining the capillary length and characteristic time,
\begin{equation}\label{eq:capillary}
l_c^{-2} = \frac{E''(R_{\text{eq}})}{2c_b(c_a-c_b)f''(c_b)}
\end{equation}
\begin{equation}\label{eq:c_time}
t_c=\frac{l_c^2c_a}{Dc_b},
\end{equation}
respectively. Here, $D$ is the diffusion coefficient.

\section{LSW theory with the quadratic Gibbs-Thomson boundary condition}
The number of solute molecules in a droplet is $c_av_dR^d$. If we assume the flux of particles into/out of the droplet at the interface is given by $\bm{j}(r=R)=-D\partial c/\partial r\lvert_{r=R}\hat{\bm{r}}$, then conservation of solute locally says
\begin{align}\label{eq:local_cons1}
\frac{d (c_av_dR^d)}{dt}&=-\int_S \bm{j}\cdot \bm{n} dS\nonumber\\
&=Da_dR^{d-1}\left.\frac{\partial c}{\partial r}\right|_{r=R}.
\end{align}

Assuming the concentration obeys steady state diffusion in the bulk $\nabla^2c=0$, the concentration field can be approximated as (in $d=3$)
\begin{equation}\label{eq:c_field}
c(r)=\bar{c}-\frac{[\bar{c}-c(R)]R}{r},
\end{equation}
where $c(R)$ is the concentration at the surface of the droplet (in equilibrium, $c(R)=c_b$), and $\bar{c}$ is the concentration far from the droplets (in equilibrium, $\bar{c}=c_b$). Inserting this into eqn \ref{eq:local_cons1} gives
\begin{equation}\label{eq:dimensional_LSW}
c_a\frac{dR}{dt}=\frac{D}{R}[\bar{c}-c(R)].
\end{equation}
If I define the supersaturation as $\chi(t)=(\bar{c}-c_b)/c_b$, and note that $c(R)-c_b=\delta c_b$, which is the Gibbs-Thomson effect of eqn \ref{eq:GT_quad_dim}, and re-write $R/l_c\to R$, $t/t_c\to t$, I get the modified LSW theory
\begin{equation}\label{eq:Rdot}
\frac{dR}{dt}=\frac{1}{R}\left[\chi(t)-(5/3R-R_{\text{eq}})(R-R_{\text{eq}})\right].
\end{equation}

\section{LSW theory with the fitted ``collagen" Gibbs-Thomson boundary condition}
In this section, I assume a ``collagen" free energy per unit volume of the form
\begin{equation}\label{eq:colFE}
E(R)\sim \frac{2\gamma}{R}-\frac{2\gamma}{\alpha R_{\text{eq}}^{1-\alpha}R^{\alpha}}+\frac{1}{2}K_{22}q^2
\end{equation}
where $\gamma$ is the surface tension between the droplet (a.k.a. fibril) phase and the surrounding solution, $K_{22}$ is the twist elastic constant, $q$ is the inverse cholesteric pitch, and $\alpha$ is a fitting parameter. We take $\alpha=1/3$ as it gives a reasonable fit to the true free energy density of collagen fibrils. Using the same methods as the previous section with $d=2$, I find that the Gibbs-Thomson boundary condition becomes
\begin{equation}\label{eq:GT_col}
\frac{\delta c_b}{c_b}=\frac{2\gamma}{c_b(c_a-c_b)f''(c_b)}\left[\frac{1}{2R}-\left(\frac{1}{\alpha}-\frac{1}{2}\right)\frac{1}{R_{\text{eq}}^{1-\alpha}R^{\alpha}}-\frac{1}{R_{\text{eq}}}\left(1-\frac{1}{\alpha}\right)\right]
\end{equation}

In this case, if I define the length scale
\begin{equation}\label{eq:length_coll}
l_c\equiv\frac{2\gamma}{c_b(c_a-c_b)f''(c_b)},
\end{equation}
and rescale $R$ and $t$ by $l_c$ and $t_c$, I get the modified LSW result
\begin{equation}\label{eq:Rdot_coll}
\frac{dR}{dt}=\frac{1}{R}\left[\chi(t)-\frac{1}{2R}+\left(\frac{1}{\alpha}-\frac{1}{2}\right)\frac{1}{R_{\text{eq}}^{1-\alpha}R^{\alpha}}+\frac{1}{R_{\text{eq}}}\left(1-\frac{1}{\alpha}\right)\right].
\end{equation}

\section{Mass conservation}
Calculation of LSW theory also requires a mass/number conservation law to determine the time evolution of the supersaturation. The total number of solute molecules in the system $N_s$ is conserved throughout the time evolution. Therefore, $N_s=N_m+N_d$, where $N_m$ is the number of solute molecules in the matrix, and $N_d$ is the number of solute molecules in the droplets. In this calculation I assume that the interfacial width of the droplets shown in Figure \ref{fig:droplet}, $\xi=0$ to simplify things.

\begin{figure}[H]
	\centering
	\includegraphics[width=0.5\textwidth]{"droplet".pdf}
	\caption{Concentration near a droplet. Inside the droplet, the concentration is $c_a$. within the interface of the droplet (the area between the inner circle and outer circle, width $\xi$) the concentration is $\delta c_b+c_b$, where $c_b$ is the equilibrium concentration. Far from the droplet, the concentration is $\bar{c}$, which at equilibrium is $c_b$. I take the sharp interface limit $\xi\sim0$ in this calculation.}
\end{figure}\label{fig:droplet}

In this $\xi=0$ limit, conservation of solute number gives
\begin{equation}\label{eq:cons1}
N_s=\bar{c}\left(V-v_d\sum_{i=1}^NR_i^d\right)+c_av_d\sum_{i=1}^NR_i^d.
\end{equation}
Rearranging this equation, taking the derivative of $\bar{c}$ with respect to $t$, and noting $\partial\bar{c}/\partial t = c_b\partial \chi(t)/\partial t$, I get
\begin{equation}\label{eq:cons2}
c_b\frac{\partial\chi}{\partial t}=\left[\frac{-c_aV+N_s}{\left(V-v_d\sum_{i=1}^NR_i^d\right)^2}\right]dv_d\sum_{i=1}^NR_i^{d-1}\frac{\partial R_i}{\partial t}.
\end{equation}
Defining $\alpha\equiv N_s/(Vc_a)$, $\beta=c_a/c_b$, and rescaling the length and time of the system by $l_c$ and $t_c$ respectively, the evolution equation of the supersaturation is
\begin{equation}\label{eq:ss_evolution}
\frac{\partial\chi}{\partial t}=\beta\frac{(\alpha-1)}{\left(1-\frac{v_d}{V}\sum_{i=1}^NR_i^d\right)^2}\frac{dv_d}{V}\sum_{i=1}^NR_i^{d-1}\frac{\partial R_i}{\partial t}.
\end{equation}
Since we are assuming that in equilibrium, the solute is more concentrated in the minority phase, the parameters $\alpha<1$ and $\beta>1$. $\alpha\geq1$ corresponds to a solution starting with $\bar{c}\geq c_a$, which would physically correspond to a supersaturation too high to maintain droplets at concentration $c_a$ (i.e. the whole system would become a supersaturated "droplet" phase). With the definitions $\alpha$ and $\beta$, the conservation law eqn \ref{eq:cons1} becomes in dimensionless form
\begin{equation}\label{eq:rescaled_cons}
\alpha=\frac{(1+\chi(t))}{\beta}\left(1-\frac{v_d}{V}\sum_{i=1}^NR_i^d\right)+\frac{v_d}{V}\sum_{i=1}^NR_i^d.
\end{equation}
This can also be rearranged to specify the initial fraction of the system volume which starts in droplet form, if I define
\begin{equation}\label{eq:minority_fraction}
\phi_0=\frac{\alpha-\frac{(1+\chi)}{\beta}}{1-\frac{(1+\chi)}{\beta}}
\end{equation}
as the initial volume fraction of the minority phase.
\section{Multi-droplet equation (non-zero volume fraction limit)}
For the case of multiple droplets, the PDE
\begin{equation}\label{eq:dimensional_diff}
\nabla^2c=a_d\sum_{i=1}^NQ_i\delta(\bm{r}-\bm{r}_i)
\end{equation}
describes the diffusion field, treating each droplet as a source or sink with strength $Q_i$. Similar to the single droplet case of eqn \ref{eq:local_cons1}, using fick's law $\bm{j}=-D\nabla c$ and the divergence theorem, I can write the local conservation law as
\begin{equation}\label{eq:local_cons2}
R_i^{d-1}\frac{d R_i}{d t}=\frac{D}{c_a}Q_i.
\end{equation}
If I define $\theta(\tilde{\bm{r}},\tilde{t})=(c(\bm{r},t)-c_b)/c_b$ ($\tilde{r}=r/l_c$, $\tilde{t}=t/t_c$), and re-write eqn \ref{eq:dimensional_diff} in dimensionless form, I get the dimensionless diffusion equation
\begin{equation}\label{eq:dimensionless_diff}
\nabla^2\theta=a_d\sum_{i=1}^NB_i\delta(\bm{r}-\bm{r}_i)
\end{equation}
where I have written everything in terms of dimensionless variables (e.g. $r/l_c\to r$), and $B_i = Q_il_c^{2-d}/c_b$. Therefore, in dimensionless form the local conservation law becomes
\begin{equation}\label{eq:local_cons3}
R_i^{d-1}\frac{dR_i}{dt}=B_i.
\end{equation}
Note that with this result, the supersaturation evolution equation becomes
\begin{equation}\label{eq:ss_evolution2}
\frac{\partial\chi}{\partial t}=\beta\frac{(\alpha-1)}{\left(1-\frac{v_d}{V}\sum_{i=1}^NR_i^d\right)^2}\frac{dv_d}{V}\sum_{i=1}^NB_i
\end{equation}
Solving eqn \ref{eq:dimensionless_diff} with the condition $\lim_{r\to\infty}\theta(\bm{r},t)=\chi(t)$, introducing a monopole approximation, and implementing the quadratic Gibbs-Thomson boundary condition (eqn \ref{eq:GT_quad_dim}), I get the equation (in $d=3$)
\begin{equation}\label{eq:system_of_eqns_Bi}
\left(\frac{5}{3}R_j-R_{\text{eq}}\right)(R_j-R_{\text{eq}})=\chi(t)-\frac{B_j}{R_j}-\sum_{i\neq j}^N\frac{B_i}{\left|\bm{r}_j-\bm{r}_i\right|}.
\end{equation}
In $d=2$, the equations are of the form
\begin{equation}\label{eq:system_of_eqns_Bi_2d}
\left(2R_j-R_{\text{eq}}\right)(R_j-R_{\text{eq}})=\chi(t)+B_j\log\left(\frac{R_j}{L}\right)+\sum_{i\neq j}^NB_i\log\left(\frac{\left|\bm{r}_j-\bm{r}_i\right|}{L}\right).
\end{equation}

\section{Simulation details}
In order to simulate the growth/shrinking of these droplets, I start by specifying the distribution of radii $\{R_i\}$, and the initial supersaturation $\chi(0)$. I also have two more degrees of freedom, $\alpha=N_s/(Vc_a)<1$ and $\beta=c_a/c_b>1$. Note that there is a limited range of $\chi(0)$ values which can be specified before $\alpha$ becomes too large (i.e. larger than 1). The inequality $\beta>1+\chi(t)$ must be satisfied to prevent this from occurring. For simplicity of interpretation, I will choose also to work with the initial minority volume fraction $\phi_0$ of the droplet phase and the ratio of concentrations within the droplet and equilibrium bulk phases $\beta$ ($\alpha$ can then be determined via eqn \ref{eq:minority_fraction}). I will take $d=3$ below, but the same thing applies to the $d=2$ case, just using eqn \ref{eq:system_of_eqns_Bi_2d} instead of eqn \ref{eq:system_of_eqns_Bi}.

\subsection{numerical scheme 1}
Given the initial values $R_i$ and $\chi(0)$, I can calculate the $B_i$ of eqn \ref{eq:system_of_eqns_Bi} by inverting an $(N+1)\times (N+1)$ matrix, which is given by
\begin{align}\label{eq:matrix_Nside}
\begin{bmatrix}
\left(\frac{5}{3}R_1-R_{\text{eq}}\right)(R_1-R_{\text{eq}}) \\
\left(\frac{5}{3}R_2-R_{\text{eq}}\right)(R_2-R_{\text{eq}}) \\
\left(\frac{5}{3}R_3-R_{\text{eq}}\right)(R_3-R_{\text{eq}}) \\
\cdot \\
\cdot \\
\chi(t-dt) 
\end{bmatrix}=
\begin{bmatrix}
\frac{-1}{R_1} & \frac{-1}{|\bm{r}_1-\bm{r}_2|} & \frac{-1}{|\bm{r}_1-\bm{r}_3|} & \cdots & 1 \\
\frac{-1}{|\bm{r}_1-\bm{r}_2|} & \frac{-1}{R_2} & \frac{-1}{|\bm{r}_2-\bm{r}_3|} & \cdots & 1 \\
\frac{-1}{|\bm{r}_1-\bm{r}_3|} & \frac{-1}{|\bm{r}_2-\bm{r}_3|} & \frac{-1}{R_3} & \cdots & 1 \\
\cdot & \cdot &\cdot & \cdots & \cdot \\
\cdot & \cdot &\cdot & \cdots & \cdot \\
-dt\mathcal{A} & -dt\mathcal{A} & \cdots & \cdots & 1
\end{bmatrix}
\begin{bmatrix}
B_1 \\
B_2 \\
B_3 \\
\cdot \\
\cdot \\
\chi(t)
\end{bmatrix}.
\end{align}
Here, I have included the supersaturation calculation at time $t$ within the calculation of the $B_i$ values (through discretization of eqn \ref{eq:ss_evolution}). $\mathcal{A}$ is the right hand side of eqn \ref{eq:ss_evolution2} without the summation term. Note that the $dt$ value here is the last time step value, and so $dt=0$ at $t=0$.

Once the values of the source and sink terms $B_i$ are calculated, the smallest time step at which only a single droplet shrinks is calculated by discretizing eqn \ref{eq:local_cons3}, and setting $R_i^d(t+dt)=0$, and then chosen if it is below some threshold time step $dt_{\text{max}}$, i.e.
\begin{equation}\label{eq:min_step}
dt=\min_{i\:s.t.\: B_i<0}\left\{\frac{-R_i^d}{B_i d},dt_{\text{max}}\right\}.
\end{equation}
Finally, the radii are updated using eqn \ref{eq:local_cons3}. This process is then repeated.

\subsection{numerical schemes 2a and 2b}

Given the initial values $R_i$ and $\chi(0)$, I can calculate the $B_i$ of eqn \ref{eq:system_of_eqns_Bi} by inverting an $N\times N$ matrix, which is given by
\begin{align}\label{eq:matrix_Nside}
\begin{bmatrix}
\left(\frac{5}{3}R_1-R_{\text{eq}}\right)(R_1-R_{\text{eq}})-\chi(t) \\
\left(\frac{5}{3}R_2-R_{\text{eq}}\right)(R_2-R_{\text{eq}})-\chi(t) \\
\left(\frac{5}{3}R_3-R_{\text{eq}}\right)(R_3-R_{\text{eq}})-\chi(t) \\
\cdot \\
\cdot \\
\left(\frac{5}{3}R_N-R_{\text{eq}}\right)(R_N-R_{\text{eq}})-\chi(t) \\
\end{bmatrix}=
\begin{bmatrix}
\frac{-1}{R_1} & \frac{-1}{|\bm{r}_1-\bm{r}_2|} & \frac{-1}{|\bm{r}_1-\bm{r}_3|} & \cdots & \frac{-1}{|\bm{r}_1-\bm{r}_N|}\\
\frac{-1}{|\bm{r}_1-\bm{r}_2|} & \frac{-1}{R_2} & \frac{-1}{|\bm{r}_2-\bm{r}_3|} & \cdots & \frac{-1}{|\bm{r}_2-\bm{r}_N|}\\
\frac{-1}{|\bm{r}_1-\bm{r}_3|} & \frac{-1}{|\bm{r}_2-\bm{r}_3|} & \frac{-1}{R_3} & \cdots  & \frac{-1}{|\bm{r}_3-\bm{r}_N|}\\
\cdot & \cdot &\cdot & \cdots & \cdot \\
\cdot & \cdot &\cdot & \cdots & \cdot \\
\frac{-1}{|\bm{r}_1-\bm{r}_N|} & \frac{-1}{|\bm{r}_2-\bm{r}_N|} & \frac{-1}{|\bm{r}_3-\bm{r}_N|} & \cdots  & \frac{-1}{R_N}\\
\end{bmatrix}
\begin{bmatrix}
B_1 \\
B_2 \\
B_3 \\
\cdot \\
\cdot \\
B_N
\end{bmatrix}.
\end{align}
Here, I have included the supersaturation calculation at time $t$ within the calculation of the $B_i$ values (through discretization of eqn \ref{eq:ss_evolution}), which should allow for conservation of solute regardless of time step size. $\mathcal{A}$ is the right hand side of eqn \ref{eq:ss_evolution} without the summation term. Note that the $dt$ value here is the last time step value, and so $dt=0$ at $t=0$.

\subsubsection{2a}
Once the values of the source and sink terms $B_i$ are calculated, the smallest time step at which only a single droplet shrinks is calculated by discretizing eqn \ref{eq:local_cons3}, and setting $R_i^d(t+dt)=0$, and then chosen if it is below some threshold time step $dt_{\text{max}}$, i.e.
\begin{equation}\label{eq:min_step}
dt=\min_{i\:s.t.\: B_i<0}\left\{\frac{-R_i^d}{B_i d},dt_{\text{max}}\right\}.
\end{equation}
Finally, the supersaturation and radii are updated using eqns \ref{eq:local_cons3} and \ref{eq:ss_evolution2}, respectively. This process is then repeated.

\subsubsection{2b}
Once the values of the source and sink terms $B_i$ are calculated, the smallest time step at which only a single droplet shrinks is calculated by discretizing eqn \ref{eq:local_cons3}, and setting $R_i^d(t+dt)=0$, and then chosen if it is below some threshold time step $dt_{\text{max}}$, i.e.
\begin{equation}\label{eq:min_step}
dt=\min_{i\:s.t.\: B_i<0}\left\{\frac{-R_i^d}{B_i d},dt_{\text{max}}\right\}.
\end{equation}
The radii are updated using eqns \ref{eq:local_cons3}. Finally, the supersaturation is updated using eqn \ref{eq:rescaled_cons}. This process is then repeated.


%%%%%%%%%%%%%%%
% Bibliography
%%%%%%%%%%%%%%%
\clearpage
\bibliography{modified_GT}
\bibliographystyle{unsrt}

\end{document}
