\documentclass[12pt]{article}
%\usepackage{helvet}
%\renewcommand{\familydefault}{\sfdefault}
\usepackage{amsfonts}
\usepackage{amsmath}
\usepackage{amssymb}
\usepackage{bm}
\usepackage{fullpage}
\usepackage{setspace}
\usepackage{graphicx}
\usepackage{gensymb}
\usepackage[nottoc,numbib]{tocbibind}
\usepackage{graphicx}
\usepackage{float}
\usepackage{braket}
\usepackage{titlesec}
\usepackage{siunitx}
\usepackage{mathtools}
\usepackage{tikz}
\usepackage[font={small}]{caption}
\usepackage{subcaption}
%\titlespacing*{\section}{0pt}{4pt}{4pt}
\usepackage[letterpaper, margin=2cm]{geometry}
\newcommand*\diff{\mathop{}\!\mathrm{d}}
\DeclareSIUnit\Molar{\textsc{m}}
\DeclareMathOperator*{\MAXVAL}{max}
\DeclarePairedDelimiter\abs{\lvert}{\rvert}
\tikzstyle{place}=[circle,draw,fill,thick,inner sep = 0pt,minimum size = 1.5mm]
\tikzstyle{theline}=[line width = 1pt]


\begin{document}
\pagenumbering{arabic}
\spacing{1.5}

%%%%%%%%%%%
% Begin Document
%%%%%%%%%%%
\section{Introduction}
The theory of droplet coarsening was first developed over 60 years ago, through the seminal work of Lifshitz, Slyozov, and Wagner (cite). Motivated by early experimental observations of liquid-liquid phase separation (cite), these scientists developed what is now known as Lifshitz-Slyozov-Wagner (LSW) theory, which describes the dynamics of droplet growth in late-stage phase separation of a two-component liquid. The main result of LSW theory was to show that the average droplet size of the minority (i.e. low volume fraction) phase grows according to the scaling law $<R>\sim t^{\frac{1}{3}}$, where $<R>$ is the average droplet radius. This generic behaviour is known as coarsening or ``Ostwald ripening'', after the person who discovered it experimentally?? In the years following the basic theory, generalizations were introduced to allow for d-dimensional systems (cite) at non-zero volume fractions of minority droplet phase (cite), and a mathematical framework has been developed to determine the (time-dependent) behaviour of the radius distribution underlying $<R>$ (cite).

Recently, there has been a peaked interest in achieving steady state, mono-disperse droplets with specified radius for industrial applications. Without modification, Ostwald ripening is incompatible with mono-disperse, steady-state droplets (as can clearly be seen by the growth law $<R>\sim t^{\frac{1}{3}}$). Introduction of new physics including long-range interactions between particles (cite), chemical stabilizers which form emulsions (? cite), and ??? (cite) have shown promise in creating these mono-disperse morphologies. In this paper, we demonstrate that there is a further method of size control available to droplets with internal structure, which we refer to as equilibrium size control.

We hypothesize that equilibrium size control arises in systems with a preferred, equilibrium droplet radius $R_{\mathrm{eq}}$ which arises from short-ranged, intra-droplet internal structure, in contrast to long-range interactions mentioned above (cite). Such systems seem to be found pre-dominantly within biology, e.g. liquid-liquid phase separation between the cell nucleus and its surrounding cytoplasm, collagen fibrils (when being considered in the 2-D plane), as well as the finite-size of the cell itself. At first glance, such systems may be considered to be boring dynamically, as if one waits long enough, the droplets should approach equilibrium size, and so mono-dispersity is more or less achieved, but size-control is no longer viable. However, we show here that the approach to equilibrium is highly dependent on the supersaturation of the inter-droplet matrix, and so it is possible to reach steady state at $<R>\neq R_{\mathrm{eq}}$ if one can control the initial supersaturation and distribution of droplet sizes.


\section{Results}

Plots:
1) Flow diagram in chi(0) R(0) space.
2) Number of drops in steady state at chi(0) R(0) space
3) Maybe distribution splitting at unstable fixed point?
4) Maybe demonstrate the collapse of the initial distribution into a delta function?

Calculations:
1) Gibbs-Thomson
2) Phase-plane (Rdot vs R)
3) Evolution of droplet distribution??

\section{Conclusions}
Able to control droplet size of drops with preferred radius, and study dynamics of droplet growth using simple coarsening.

%%%%%%%%%%%%%%%
% Bibliography
%%%%%%%%%%%%%%%
\clearpage
\bibliography{modified_GT}
\bibliographystyle{unsrt}

\end{document}
