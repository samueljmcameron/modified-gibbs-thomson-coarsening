\documentclass[12pt]{article}
%\usepackage{helvet}
%\renewcommand{\familydefault}{\sfdefault}
\usepackage{amsfonts}
\usepackage{amsmath}
\usepackage{amssymb}
\usepackage{bm}
\usepackage{fullpage}
\usepackage{setspace}
\usepackage{graphicx}
\usepackage{gensymb}
\usepackage[nottoc,numbib]{tocbibind}
\usepackage{graphicx}
\usepackage{float}
\usepackage{braket}
\usepackage{titlesec}
\usepackage{siunitx}
\usepackage{mathtools}
\usepackage{tikz}
\usepackage[font={small}]{caption}
\usepackage{subcaption}
%\titlespacing*{\section}{0pt}{4pt}{4pt}
\usepackage[letterpaper, margin=2cm]{geometry}
\newcommand*\diff{\mathop{}\!\mathrm{d}}
\DeclareSIUnit\Molar{\textsc{m}}
\DeclareMathOperator*{\MAXVAL}{max}
\DeclarePairedDelimiter\abs{\lvert}{\rvert}
\tikzstyle{place}=[circle,draw,fill,thick,inner sep = 0pt,minimum size = 1.5mm]
\tikzstyle{theline}=[line width = 1pt]


\begin{document}
\pagenumbering{arabic}
\spacing{1.5}

%%%%%%%%%%%
% Begin Document
%%%%%%%%%%%
\section{Intro}
Controlling droplet size motivations:
1) Polymer dispersed liquid crystals (see ref. \cite{ohta:2012gd}). These are thin film devices where liquid crystal is in a polymer matrix, and the droplets are trying to be grown to a specific size.
2) Controlling the size of LC droplets for Biosensors (see ref. \cite{Lee:2016hd}).

Monodispersity is important in many other applications. So it would be ideal to be able to tune the droplet parameters (droplet number, droplet size).

In this paper, we demonstrate that for any system containing droplets of preferred radii, it is possible to tune the radius of droplets through control of the supersaturation (how does one experimentally control supersaturation??) and initial droplet distribution (any way to control this experimentally??). Furthermore, these two parameters can also predict the number of droplets which remain (i.e. don't evaporate) in steady-state.

Questions/things to put in intro:
1) What other systems require mono-disperse droplets?
2) How to control supersaturation experimentally, and initial radius distribution?
3) Examples of systems with a preferred radius (e.g. collagen), maybe spherulites?
4) What about nucleation??

\section{Results}

Plots:
1) Flow diagram in chi(0) R(0) space.
2) Number of drops in steady state at chi(0) R(0) space
3) Maybe distribution splitting at unstable fixed point?
4) Maybe demonstrate the collapse of the initial distribution into a delta function?

Calculations:
1) Gibbs-Thomson
2) Phase-plane (Rdot vs R)
3) Evolution of droplet distribution??

\section{Conclusions}
Able to control droplet size of drops with preferred radius, and study dynamics of droplet growth using simple coarsening.
%%%%%%%%%%%%%%%
% Bibliography
%%%%%%%%%%%%%%%
\clearpage
\bibliography{modified_GT}
\bibliographystyle{unsrt}

\end{document}
